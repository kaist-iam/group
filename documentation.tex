% Options for packages loaded elsewhere
\PassOptionsToPackage{unicode}{hyperref}
\PassOptionsToPackage{hyphens}{url}
\PassOptionsToPackage{dvipsnames,svgnames,x11names}{xcolor}
%
\documentclass[
  letterpaper,
  DIV=11,
  numbers=noendperiod]{scrartcl}

\usepackage{amsmath,amssymb}
\usepackage{lmodern}
\usepackage{iftex}
\ifPDFTeX
  \usepackage[T1]{fontenc}
  \usepackage[utf8]{inputenc}
  \usepackage{textcomp} % provide euro and other symbols
\else % if luatex or xetex
  \usepackage{unicode-math}
  \defaultfontfeatures{Scale=MatchLowercase}
  \defaultfontfeatures[\rmfamily]{Ligatures=TeX,Scale=1}
\fi
% Use upquote if available, for straight quotes in verbatim environments
\IfFileExists{upquote.sty}{\usepackage{upquote}}{}
\IfFileExists{microtype.sty}{% use microtype if available
  \usepackage[]{microtype}
  \UseMicrotypeSet[protrusion]{basicmath} % disable protrusion for tt fonts
}{}
\makeatletter
\@ifundefined{KOMAClassName}{% if non-KOMA class
  \IfFileExists{parskip.sty}{%
    \usepackage{parskip}
  }{% else
    \setlength{\parindent}{0pt}
    \setlength{\parskip}{6pt plus 2pt minus 1pt}}
}{% if KOMA class
  \KOMAoptions{parskip=half}}
\makeatother
\usepackage{xcolor}
\setlength{\emergencystretch}{3em} % prevent overfull lines
\setcounter{secnumdepth}{-\maxdimen} % remove section numbering
% Make \paragraph and \subparagraph free-standing
\ifx\paragraph\undefined\else
  \let\oldparagraph\paragraph
  \renewcommand{\paragraph}[1]{\oldparagraph{#1}\mbox{}}
\fi
\ifx\subparagraph\undefined\else
  \let\oldsubparagraph\subparagraph
  \renewcommand{\subparagraph}[1]{\oldsubparagraph{#1}\mbox{}}
\fi


\providecommand{\tightlist}{%
  \setlength{\itemsep}{0pt}\setlength{\parskip}{0pt}}\usepackage{longtable,booktabs,array}
\usepackage{calc} % for calculating minipage widths
% Correct order of tables after \paragraph or \subparagraph
\usepackage{etoolbox}
\makeatletter
\patchcmd\longtable{\par}{\if@noskipsec\mbox{}\fi\par}{}{}
\makeatother
% Allow footnotes in longtable head/foot
\IfFileExists{footnotehyper.sty}{\usepackage{footnotehyper}}{\usepackage{footnote}}
\makesavenoteenv{longtable}
\usepackage{graphicx}
\makeatletter
\def\maxwidth{\ifdim\Gin@nat@width>\linewidth\linewidth\else\Gin@nat@width\fi}
\def\maxheight{\ifdim\Gin@nat@height>\textheight\textheight\else\Gin@nat@height\fi}
\makeatother
% Scale images if necessary, so that they will not overflow the page
% margins by default, and it is still possible to overwrite the defaults
% using explicit options in \includegraphics[width, height, ...]{}
\setkeys{Gin}{width=\maxwidth,height=\maxheight,keepaspectratio}
% Set default figure placement to htbp
\makeatletter
\def\fps@figure{htbp}
\makeatother

\KOMAoption{captions}{tableheading}
\makeatletter
\@ifpackageloaded{tcolorbox}{}{\usepackage[many]{tcolorbox}}
\@ifpackageloaded{fontawesome5}{}{\usepackage{fontawesome5}}
\definecolor{quarto-callout-color}{HTML}{909090}
\definecolor{quarto-callout-note-color}{HTML}{0758E5}
\definecolor{quarto-callout-important-color}{HTML}{CC1914}
\definecolor{quarto-callout-warning-color}{HTML}{EB9113}
\definecolor{quarto-callout-tip-color}{HTML}{00A047}
\definecolor{quarto-callout-caution-color}{HTML}{FC5300}
\definecolor{quarto-callout-color-frame}{HTML}{acacac}
\definecolor{quarto-callout-note-color-frame}{HTML}{4582ec}
\definecolor{quarto-callout-important-color-frame}{HTML}{d9534f}
\definecolor{quarto-callout-warning-color-frame}{HTML}{f0ad4e}
\definecolor{quarto-callout-tip-color-frame}{HTML}{02b875}
\definecolor{quarto-callout-caution-color-frame}{HTML}{fd7e14}
\makeatother
\makeatletter
\makeatother
\makeatletter
\makeatother
\makeatletter
\@ifpackageloaded{caption}{}{\usepackage{caption}}
\AtBeginDocument{%
\ifdefined\contentsname
  \renewcommand*\contentsname{Table of contents}
\else
  \newcommand\contentsname{Table of contents}
\fi
\ifdefined\listfigurename
  \renewcommand*\listfigurename{List of Figures}
\else
  \newcommand\listfigurename{List of Figures}
\fi
\ifdefined\listtablename
  \renewcommand*\listtablename{List of Tables}
\else
  \newcommand\listtablename{List of Tables}
\fi
\ifdefined\figurename
  \renewcommand*\figurename{Figure}
\else
  \newcommand\figurename{Figure}
\fi
\ifdefined\tablename
  \renewcommand*\tablename{Table}
\else
  \newcommand\tablename{Table}
\fi
}
\@ifpackageloaded{float}{}{\usepackage{float}}
\floatstyle{ruled}
\@ifundefined{c@chapter}{\newfloat{codelisting}{h}{lop}}{\newfloat{codelisting}{h}{lop}[chapter]}
\floatname{codelisting}{Listing}
\newcommand*\listoflistings{\listof{codelisting}{List of Listings}}
\makeatother
\makeatletter
\@ifpackageloaded{caption}{}{\usepackage{caption}}
\@ifpackageloaded{subcaption}{}{\usepackage{subcaption}}
\makeatother
\makeatletter
\@ifpackageloaded{tcolorbox}{}{\usepackage[many]{tcolorbox}}
\makeatother
\makeatletter
\@ifundefined{shadecolor}{\definecolor{shadecolor}{rgb}{.97, .97, .97}}
\makeatother
\makeatletter
\makeatother
\ifLuaTeX
  \usepackage{selnolig}  % disable illegal ligatures
\fi
\IfFileExists{bookmark.sty}{\usepackage{bookmark}}{\usepackage{hyperref}}
\IfFileExists{xurl.sty}{\usepackage{xurl}}{} % add URL line breaks if available
\urlstyle{same} % disable monospaced font for URLs
\hypersetup{
  pdftitle={Documentation},
  pdfauthor={IAM team},
  colorlinks=true,
  linkcolor={blue},
  filecolor={Maroon},
  citecolor={Blue},
  urlcolor={Blue},
  pdfcreator={LaTeX via pandoc}}

\title{Documentation}
\author{IAM team}
\date{8/23/23}

\begin{document}
\maketitle
\ifdefined\Shaded\renewenvironment{Shaded}{\begin{tcolorbox}[frame hidden, boxrule=0pt, borderline west={3pt}{0pt}{shadecolor}, breakable, enhanced, sharp corners, interior hidden]}{\end{tcolorbox}}\fi

\hypertarget{living-document}{%
\section{Living Document}\label{living-document}}

\hypertarget{prerequisites}{%
\subsection{1. Prerequisites}\label{prerequisites}}

\begin{itemize}
\tightlist
\item
  \href{https://github.com/GCAM-KAIST/gcam-core}{Github}: Sign up for
  github account and email me to give you access to GCAM-KAIST
\item
  Git client: download a git client of your choice.

  \begin{itemize}
  \tightlist
  \item
    \href{https://tortoisegit.org/}{TortoiseGit} : old school client
  \item
    \href{https://desktop.github.com/}{Github desktop} : standard github
    client
  \item
    \href{https://gitforwindows.org/}{git for windows} : command line
    based professional git client
  \end{itemize}
\item
  \href{https://symbolclick.com/xmlmarker_1_1_setup.exe}{XML marker
  1.1}: make sure you get the older 1.1 version
\item
  \href{https://posit.co/download/rstudio-desktop/}{R}
\end{itemize}

\hypertarget{useful-instruction}{%
\subsection{2. Useful Instruction}\label{useful-instruction}}

\begin{itemize}
\tightlist
\item
  \href{https://jgcri.github.io/gcam_training/gcam.html}{GCAM tutorial}
\item
  \href{https://www.jrebel.com/blog/git-cheat-sheet}{Git}
\item
  \href{https://jgcri.github.io/gcam-doc/user-guide.html}{GCAM}
\item
  \href{https://jgcri.github.io/gcamdata/articles/getting-started/getting-started.html}{gcamdata}
\item
  Useful guide/PPT slides from
  \href{https://github.com/JGCRI/gcam_training/tree/main/presentations}{gcam-training}
\item
  videos
\end{itemize}

\hypertarget{running-gcam-release}{%
\subsection{3. Running GCAM release}\label{running-gcam-release}}

\begin{enumerate}
\def\labelenumi{\arabic{enumi}.}
\tightlist
\item
  \href{https://jgcri.github.io/gcam-doc/user-guide.html}{GCAM}
\item
  Download
  \href{https://github.com/JGCRI/gcam-core/releases/tag/gcam-v7.0}{standard
  GCAM release}
\item
  Set up some prerequisite apps and run!
\end{enumerate}

\hypertarget{making-your-own-git-branch}{%
\subsection{4. Making your own git
branch}\label{making-your-own-git-branch}}

\begin{tcolorbox}[enhanced jigsaw, leftrule=.75mm, opacitybacktitle=0.6, arc=.35mm, breakable, opacityback=0, left=2mm, coltitle=black, toptitle=1mm, toprule=.15mm, bottomtitle=1mm, bottomrule=.15mm, titlerule=0mm, title=\textcolor{quarto-callout-note-color}{\faInfo}\hspace{0.5em}{Note}, rightrule=.15mm, colback=white, colbacktitle=quarto-callout-note-color!10!white, colframe=quarto-callout-note-color-frame]
if anything on github is confusing, just ask chatGPT!
\end{tcolorbox}

\begin{enumerate}
\def\labelenumi{\arabic{enumi}.}
\item
  Go to \href{https://github.com/GCAM-KAIST/gcam-core}{GCAM-KAIST}.
\item
  Pick a branch to start from (currently: hcm/proj/kaist7update).
\item
  Create your own branch {[}you own this. No one else will make changes
  to your own branch{]}.
\item
  \href{https://github.com/GCAM-KAIST/gcam-core.git}{git clone} {[}this
  creates a full repo of GCAM-KAIST{]}.
\item
  git checkout {[}your branch name{]} {[}this sets your branch on your
  own computer{]}.
\item
  Try making a small inconsequential change. Such as adding a small xml
  file in the input/policy folder.
\item
  git add {[}file you added{]}.
\item
  git commit {[}this will commit all files you added/changed{]}.
\item
  git push.
\item
  Confirm that your commit has been correctly pushed to github.
\item
  (if you are happy with your development, and would like it to be
  applied to ALL GCAM-KAIST users) create a pull request (PR) {[}back
  into the main GCAM-KAIST repo{]}.
\item
  Assign me (d3y419) as the reviewer. I'll review the PR and merge as
  needed.
\end{enumerate}

\hypertarget{transition-to-gcam-kaist-7.0-instructions-and-next-steps}{%
\section{Transition to GCAM-KAIST 7.0: Instructions and Next
Steps}\label{transition-to-gcam-kaist-7.0-instructions-and-next-steps}}

As we embark on an exciting and crucial phase of updating our energy
assessment model from GCAM-KAIST 2.0 to GCAM-KAIST 7.0, the following
instructions and guidelines will ensure a smooth transition. Each one of
the team is responsible for a particular XML file which is identified in
the following \href{gcam-kaist-upgrade7b.xlsx}{link}.

\hypertarget{download-the-required-versions}{%
\subsection{1. Download the Required
Versions}\label{download-the-required-versions}}

GCAM 7.0 Official Release by JCRI: You can download this latest version
from the following
\href{https://github.com/JGCRI/gcam-core/releases/download/gcam-v7.0/gcam-v7.0-Windows-Release-Package.zip}{link}.

GCAM-KAIST 2.0: This is our specialized version at KAIST. Please
download it using the
\href{https://o365kaist-my.sharepoint.com/:f:/g/personal/ahmeds_office_kaist_ac_kr/EuCnuJZbojFLjz5o0q3dwaYBQ4QihxB_7z8umGiHIeos8A?e=jnZFqE}{link}
.

\hypertarget{analyze-the-differences}{%
\subsection{2. Analyze the Differences}\label{analyze-the-differences}}

XML File Comparison: Please compare the XML files between the two
versions as per your specific task in the google sheet. Document the
Differences: Note any discrepancies, changes, or new features. This will
help us understand what needs to be transferred or adapted.

\hypertarget{update-gcam-7.0-with-gcam-kaist-2.0-data}{%
\subsection{3. Update GCAM 7.0 with GCAM-KAIST 2.0
Data}\label{update-gcam-7.0-with-gcam-kaist-2.0-data}}

Modify the Original GCAM 7.0 XML Files: Replace the necessary data with
the corresponding data from GCAM-KAIST 2.0. Follow the Evolution
Process: This is vital to complete the transformation from GCAM-KAIST
2.0 to GCAM-KAIST 7.0 by checking the ``comparison with GCAM-KAIST2''
column in google sheet. After running the scenario, you can check if the
results are different from GCAM-KAIST2.0 or not.

\hypertarget{prepare-for-tomorrows-meeting}{%
\subsection{4. Prepare for Tomorrow's
Meeting}\label{prepare-for-tomorrows-meeting}}

Clarification of Differences: Be prepared to discuss the differences
you've identified and how you've addressed them.



\end{document}
